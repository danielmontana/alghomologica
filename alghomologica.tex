\documentclass[a4paper,12pt,leqno]{report}
\input{comandos}
\begin{document}
\chapter*{Introducción al álgebra homológica}
\section*{Módulos}

La idea de módulo se puede ver como una generalización de los espacios vectoriales y los grupos abelianos:
\begin{definicion}
Sea $R$ un anillo. Un $R$-módulo a la izquierda es un grupo abeliano $A$ con una acción $R \times A \rightarrow A$ tal que para todo $r,r'\in R,a_1,a_2\in A$:
\begin{enumerate}
	\item $r(a_1+a_2)=r(a_1)+r(a_2),$
\end{enumerate}

\end{definicion}





\end{document}          
